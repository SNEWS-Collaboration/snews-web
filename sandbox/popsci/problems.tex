\documentclass[12pt]{article}
\usepackage{amsmath,amssymb,fullpage,graphics,graphicx,yuniorlab}
\usepackage{problemset,url}

\begin{document}

\title{SNEWS Word Problems}
\author{}
\date{}
\maketitle

\section{Exercises}

\begin{probdesc}{1}{How Big is a Neutrino Detector?}
The sizes of neutrino detectors are often measured in {\em kilotons.}
One kiloton is a thousand tons or one million kilograms, so a
ten-kiloton water-based detector contains one million kilograms of
water in its tank (when it is completely filled).  (The infamous
sea-going vessel HMS {\em Titanic} had a mass of about 26 kilotons.)
For the following, it may be useful to know that the density of water
is about 1 gram per cubic centimeter, or 1 kilogram per liter.  Also,
according to the way the metric system was defined, it takes 1000
($10^3$) liters to make up one cubic meter.

\begin{enumerate}
\item[(a)] Suppose we want to build a 50-kiloton water-based neutrino
  detector.  As a first estimate to the size our tank must be, imagine
  that the tank will be built in the shape of a cube.  How many meters
  long must the side of the cube be to hold 50 kilotons of H$_2$O when
  it is completely filled?

\end{enumerate}

\end{probdesc}

\begin{probdesc}{2}{How Strong is the Signal?}
We measure {\em intensity} in terms of the energy falling on a unit
area, like one square meter, in a small unit of time, like one
second.
\begin{displaymath}
{\rm intensity} = \frac{\rm amount\ of\ energy}{({\rm area})({\rm
    time})}.
\end{displaymath}
As we move away from a source of light, sound or other ``radiant
energy'', it becomes dimmer.  The {\em inverse square law} says that
intensity drops off with the square of the distance:  if the source
moves to twice its original distance, it appears four times as dim.
We can write this rule as an equation; if we measure an intensity
$I_0$ at an original distance $R_0$, then the intensity at some other
radius $R$ is given by
\begin{equation}
I(R) = I_0 \left(\frac{R_0}{R}\right)^2.
\end{equation}

\begin{enumerate}
\item[(a)] Suppose that a certain lamp radiates 1000 joules (1
  kilojoule, or 1 kJ) of energy per square meter per second, when
  measured a distance of 10~m away.  What will the intensity be at a
  distance of 30~m?

\item[(b)] Astronomers like to find objects they can use as ``standard
  candles''.  If we know how bright a star would be at some reference
  distance, say one light-year away, and we measure how bright the
  star appears to be from Earth, we can work out how far away the star
  must be.  One type of standard candle, useful for measuring
  distances to far-away galaxies, is a Type Ia supernova.
  (Technically speaking, they aren't exactly ``standard candles'', but
  scientists have learned to correct for the differences, making them
  ``standardizeable''.  For this problem, we can pretend that all Type
  Ia supernovae have about the same intrinsic brightness.)  Suppose
  that all Type Ia supernovae emit at an intensity $I_0$ when viewed
  from a standard distance of $R_0$ = 1 megaparsec (3.26 million
  light-years).  One day, an astronomer sees a new Type Ia supernova
  in a distant galaxy, and its observed brightness is only 1\% of the
  intrinsic brightness $I_0$.  How far away is the galaxy?
\end{enumerate}
\end{probdesc}

\begin{probdesc}{3}{Temperature Scales}
Scientists like to use the {\em Kelvin scale} for measuring
temperatures.  It is convenient for many scientific purposes, because
zero on the Kelvin scale is the coldest temperature possible in the
Universe, also known as {\em absolute zero.}  Before 1967, scientists
said ``degrees Kelvin'', just as we say ``degrees Celsius'' or
``degrees Fahrenheit'', but since that year, it has been agreed to say
``kelvins'' instead.  Therefore, we say that one kelvin is the same
size as one degree Celsius.

On the Fahrenheit scale, water freezes at 32$^\circ$ and boils at
212$^\circ$, so there are 180$^\circ$ Fahrenheit between the
two points.  On the scale invented by Celsius, water freezes at
0$^\circ$ and boils at 100$^\circ$, for a difference of
100$^\circ$, meaning that there are 1.8 Fahrenheit degrees for every
degree Celsius (and for every kelvin).  Absolute zero is
$-273.15^\circ$ C, so we can write formulas for a kelvin temperature
in terms of either Fahrenheit or Celsius degrees.
\begin{equation}
K = C + 273.15 = \frac{F + 459.67}{1.8}.
\end{equation}

\begin{enumerate}
\item[(a)] What is a comfortable room temperature in degrees Celsius
  and in kelvins?  What is human body temperature in kelvins?

\item[(b)] For very large temperatures, it doesn't matter too much
  whether they are reported in Celsius degrees or in kelvins.  Georg
  Stefan estimated the surface of the Sun to have a temperature around
  5500$^\circ$C.  What is this temperature in kelvins? 
\end{enumerate}

\end{probdesc}

\begin{probdesc}{4}{Stellar Masses}
Over the last several decades, scientists have discovered that the
mass of a star is the single most important factor in determining how
long it will ``live'' and in what fashion it will ``die''.
Frequently, masses for distant stars are given as multiples of our
Sun's mass.

\begin{enumerate}
\item[(a)] Scientists believe that the star Betelgeuse (pronounced
  ``beetle-juice'') is a likely candidate to go supernova relatively
  soon---perhaps in the next few thousand years.  Betelgeuse, the red
  star marking Orion's left shoulder, is estimated to have a mass
  about 15 times that of the Sun.  What is this value in (i) kilograms
  and (ii) Earth masses?

\item[(b)] It is estimated that if the planet Jupiter were about 70
  times more massive, its core would begin fusion reactions and the
  planet would be a small star.  Estimate this mass, and give the
  result in (i) kilograms and (ii) Solar masses.  Stars in this mass
  range (below about 75 Jupiter masses) are known as {\em brown
  dwarfs.}  They do not sustain normal hydrogen fusion, although for
  the first portion of their lifespan they do fuse the heavier
  hydrogen isotope {\em deuterium.}
\end{enumerate}
\end{probdesc}

\begin{probdesc}{5}{Half-Lives}
Many phenomena relating to radioactivity can be described in terms of
half-lives.  Individual atoms decay randomly; we can't say when a
particular atom in a piece of radioactive mineral will decay, but we
can calculate how many atoms will have decayed after some amount of
time has passed.  The {\em half-life} for a substance is the amount of
time it takes for half of the material to break down into radioactive
decay products.  We use the Greek letter $\tau$ ({\em tau}) to
represent the half-life, which may be billions of years (for some
uranium isotopes) or a tiny fraction of a second.  If we begin with
$N_0$ atoms of radioactive material, the half-life equation tells us
how many atoms will still be present $t$ seconds later, a quantity we
write $N(t)$.
\begin{equation}
N(t) = N_0 \cdot \left(\frac{1}{2}\right)^{t / \tau}.
\end{equation}

\begin{enumerate}
\item[(a)] {\em Radium decay.}  As Pierre Curie first determined in
  the early years of the twentieth century, radium has a half-life of
  around sixteen centuries.  More precisely, the most stable isotope
  of radium (the one found in nature) has a half-life of 1602 years.
  This isotope, $^{226}{\rm Ra}$, decays by emitting an alpha particle (a
  helium nucleus), turning the radium atom into an atom of the
  radioactive gas radon:
  \begin{equation}
    ^{226}{\rm Ra} \rightarrow ^4{\rm He} + ^{222}{\rm Rn}.
  \end{equation}
  Suppose we begin with 10 grams of pure radium-226.  How much will be
  left after 1000 years?  If you have studied molar masses and atomic
  weights, try computing what volume of radon gas will be emitted
  during this time.

\item[(b)] {\em Carbon-14 dating.} The radioactive isotope carbon-14
  is produced by cosmic rays impacting the Earth's atmosphere.  It
  decays to nitrogen-14 with a half-life of 5,730 years.  All living
  organisms take carbon-14 into their bodies, maintaining their
  carbon-14 level at a fairly constant percentage of the total carbon
  in their systems.  Once they die, they no longer take in carbon-14,
  so radioactive decay means that the carbon-14 level will drop.
  Suppose an archaeologist finds a wooden box full of treasure buried
  in ancient Roman ruins; the gold coins in the treasure chest claim
  to be from the reign of Augustus Caesar, about two thousand years
  ago.  She takes the box to a carbon-dating lab and finds that its
  carbon-14 content is 80\% $\pm$ 2\% of what it had been when the
  trees for the box were chopped down.  Using the half-life decay
  equation, tell whether or not this measurement is consistent with
  the information from the coins.

\item[(c)] {\em Supernova light curve.}  Scientists believe that most
  of the light from a Type Ia supernova doesn't come from the original
  explosion.  Instead, it is believed that the supernova's shock wave
  creates heavy elements like uranium, which the normal nuclear fusion
  processes in stars cannot create.  Some of these elements are
  radioactive, breaking down to release secondary energy as heat and
  light.  Suppose that SNEWS catches a supernova early in its
  development, and a clever amateur astronomer takes a spectrum with
  his backyard telescope.  The spectrum reveals the presence of
  nickel-56, which has a half-life of 6.077 days.  How many days
  before the nickel-56 has decayed to $\frac{1}{64}$ of the original
  amount?

\item[(d)] {\em Bonus:} If the supernova is still half its original
  brightness 80 days after it first exploded, can nickel-56 be its
  only source of light?  If not, guess what other elements might be
  responsible.
\end{enumerate}
\end{probdesc}

\begin{probdesc}{6}{Einstein's Equation}
According to Einstein's Special Theory of Relativity, matter and
energy are interchangeable.  It is possible to convert an amount of
mass into pure energy, which may take the form of light or other
electromagnetic radiation.  The exact rule is given by Einstein's
famous equation,
\begin{equation}
E = mc^2.
\end{equation}
Here, $c$ is the speed of light, roughly $3\times10^8$ meters per
second.  If $m$ is given in kilograms and $c$ in meters per second,
then $E$ will have units of joules.
\begin{enumerate}
\item[(a)] How much energy would be released if one kilogram of matter (any
  kind of matter) were turned entirely into energy?

\item[(b)] A key characteristic of {\em antimatter} is that when it comes
  into contact with regular matter, both objects vanish, releasing
  their mass as a flood of energy.  One example of such an {\em
  annihilation} reaction happens when an electron meets its
  antiparticle, called a {\em positron} because it has the same mass
  as an electron but a positive electric charge.  Typically, when an
  electron and a positron annihilate, they produce two photons of
  equal energy.  What is the energy of each photon in MeV?  Convert
  this number to joules.

\item[(c)] According to the quantum theory Max Planck helped found,
  the wavelength of a photon is inversely proportional to its energy.
  Typically, we use the Greek letter $\lambda$ ({\em lambda}) to stand
  for the wavelength.  Planck's equation says that
  \begin{equation}
    E = \frac{hc}{\lambda}
  \end{equation}
  where $h$ is {\em Planck's constant,} a number which experiments
  show is roughly $6.626\times10^{-34}$ joule-seconds.  (Don't worry
  about the units too much for this problem.)  What is the wavelength
  of the photon whose energy you calculated in part (b)?  In which
  part of the electromagnetic spectrum does this photon fall---can you
  see it with the naked eye?

\item[(d)] Astrophysicists estimate that a Type II supernova can
  release $10^{44}$ joules of energy.  Using Einstein's equation, find
  how many kilograms lighter the supernova remnant must be than the
  original star.  Express this number as a percentage, assuming that
  the original star was twenty times as massive as the Sun.
\end{enumerate}
\end{probdesc}

\section{Useful Information}

In addition to this material, it never hurts to have a Periodic Table
handy.

\begin{table}[!h]
\begin{tabular}{cccc}
\hline\hline\\
Shape & Characteristic Length & Surface Area & Volume \\
\hline
Cube & Side $a$ & $8a^2$ & $a^3$\\
Sphere & Radius $r$ & $4\pi r^2$ & $\frac{4\pi}{3}r^3$\\
Cylinder & Height $h$, Radius $r$ & $2\pi r^2  + 2\pi rh$
& $\pi r^2 h$ \\
\hline\hline\\
\end{tabular}
\caption{Useful geometric formulas for common 3D shapes.}
\end{table}

\begin{table}[!h]
\begin{tabular}{cccccc}
\hline\hline\\
Particle & Symbol & Mass (kg) & Mass (eV/$c^2$) & Charge & Spin \\
\hline
Electron &$e^-$& $9.109\times10^{-31}$ & $5.11\times10^5$ 
  & $-1$ & $\frac{1}{2}$\\
Proton & $p$ & $1.673\times10^{-27}$ & $9.38\times10^8$ 
  & $+1$ & $\frac{1}{2}$\\
Neutron & $n$ &  $1.675\times10^{-27}$ & $9.40\times10^8$ 
  & $0$ & $\frac{1}{2}$\\
Photon & $\gamma$ & $0$ & $0$ & $0$ & $0$\\
Electron Neutrino & $\nu_e$& $\approx 0$ & $< 2.5$ & $0$ & $\frac{1}{2}$\\
Muon Neutrino & $\nu_\mu$ & $< 3\times10^{-31}$ & $< 1.7\times10^5$ 
  & $0$ & $\frac{1}{2}$\\
Tau Neutrino & $\nu_\tau$ & $< 3\times10^{-29}$ & $< 1.8\times10^7$ 
  & $0$ & $\frac{1}{2}$\\
\hline\hline\\
\end{tabular}
\caption{Basic properties of common particles.  Charges are given
in multiples of the fundamental charge unit, $1.602\times10^{-19}$ coulombs.}
\end{table}

\begin{table}[!h]
\begin{tabular}{cccc}
\hline\hline\\
Object & Mass (kg) & Radius (km) & Distance (km) \\
\hline
Earth & $5.9736\times10^{24}$ & 6,400 & --- \\
Moon & $7.348\times10^{22}$ & 1738 & 384,400\\
Jupiter & $1.899\times10^{27}$ & $71,492$ & 620 to 920 million \\
Sun & $1.9891\times10^{30}$ & $6.960\times10^5$ & $149.6$ million \\
\hline\hline\\
\end{tabular}
\caption{Masses and distances from Earth for various astronomical bodies.}
\end{table}

\begin{table}[!h]
\begin{tabular}{cccc}
\hline\hline\\
Fuel & Main Product & $T$ (10$^9$ K) & Duration (yr)\\
\hline
H & He & 0.037 & $8.1\times10^6$\\
He & O, C & 0.19 & $1.2\times10^6$\\
C & Ne, Mg & 0.87 & $9.8\times10^2$\\
O & Si, S & 2.0 & $1.3$\\
Si & Fe & 3.3 & 0.031 (11 days)\\
\hline\hline\\
\end{tabular}
\caption{Nuclear ``burning'' stages for a star of 20 solar masses.
Source: ``Massive Star Evolution Through the Ages''
({\tt http://arxiv.org/abs/astro-ph/0211062}).}
\end{table}

\end{document}
