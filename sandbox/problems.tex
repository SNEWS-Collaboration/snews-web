\documentclass[12pt]{article}
\usepackage{amsmath,amssymb,fullpage,graphics,graphicx,yuniorlab}
\usepackage{problemset,url}

\begin{document}

\title{SNEWS Word Problems}
\author{}
\date{}
\maketitle

\section{Exercises}

\begin{probdesc}{1}{How Big is a Neutrino Detector?}
The sizes of neutrino detectors are often measured in {\em kilotons.}
One kiloton is a thousand tons or one million kilograms, so a
ten-kiloton water-based detector contains one million kilograms of
water in its tank (when it is completely filled).  (The infamous
sea-going vessel HMS {\em Titanic} had a mass of about 26 kilotons.)
For the following, it may be useful to know that the density of water
is about 1 gram per cubic centimeter, or 1 kilogram per liter.  Also,
according to the way the metric system was defined, it takes 1000
($10^3$) liters to make up one cubic meter.

\begin{enumerate}
\item[(a)] Suppose we want to build a 50-kiloton water-based neutrino
  detector.  As a first estimate to the size our tank must be, imagine
  that the tank will be built in the shape of a cube.  How many meters
  long must the side of the cube be to hold 50 kilotons of H$_2$O when
  it is completely filled?

\item[(b)]
\end{enumerate}

\end{probdesc}

\begin{probdesc}{2}{How Strong is the Signal?}
\begin{equation}
I(R) = I_0 \left(\frac{R_0}{R}\right)^2.
\end{equation}
\end{probdesc}

\begin{probdesc}{3}{Temperature Scales}
\end{probdesc}

\begin{probdesc}{4}{Stellar Masses}
\end{probdesc}

\begin{probdesc}{5}{Signal Delay}
\end{probdesc}

\begin{probdesc}{6}{Nuclear Burning}
\end{probdesc}

\begin{probdesc}{7}{Half-Lives}
\begin{equation}
N(t) = N_0 \cdot \left(\frac{1}{2}\right)^{t / \tau}.
\end{equation}
\end{probdesc}

\begin{probdesc}{8}{Einstein's Equation}
According to Einstein's Special Theory of Relativity, matter and
energy are interchangeable.  It is possible to convert an amount of
mass into pure energy, which may take the form of light or other
electromagnetic radiation.  The exact rule is given by Einstein's
famous equation,
\begin{equation}
E = mc^2.
\end{equation}
Here, $c$ is the speed of light, roughly $3\times10^8$ meters per
second.
\begin{enumerate}
\item[(a)] How much energy would be released if one kilogram of matter (any
  kind of matter) were turned entirely into energy?

\item[(b)] A key characteristic of {\em antimatter} is that when it comes
  into contact with regular matter, both objects vanish, releasing
  their mass as a flood of energy.  One example of such an {\em
  annihilation} reaction happens when an electron meets its
  antiparticle, called a {\em positron} because it has the same mass
  as an electron but a positive electric charge.  Typically, when an
  electron and a positron annihilate, they produce two photons of
  equal energy.  What is the energy of each photon in MeV?  Convert
  this number to joules.

\item[(c)] According to the quantum theory Max Planck helped found, the
  wavelength of a photon is inversely proportional to its energy.
  Planck's equation says that
  \begin{equation}
    E = \frac{hc}{\lambda}
  \end{equation}
  where $h$ is {\em Planck's constant,} a number which experiments
  show is roughly $6.626\times10^{-34}$ joule-seconds.  (Don't worry
  about the units too much for this problem.)  What is the wavelength
  of the photon whose energy you calculated in part (b)?  In which
  part of the electromagnetic spectrum does this photon fall---can you
  see it with the naked eye?

\item[(d)] Astrophysicists estimate that a Type II supernova can release
  joules of energy.  Using Einstein's equation, find how many
  kilograms lighter the supernova remnant must be than the original
  star.  Express this number as a percentage, assuming that the
  original star was twenty times as massive as the Sun.
\end{enumerate}
\end{probdesc}

\section{Useful Information}

\begin{table}[!h]
\begin{tabular}{cccc}
\hline\hline\\
Shape & Characteristic Length & Surface Area & Volume \\
\hline
Cube & Side $a$ & $8a^2$ & $a^3$\\
Sphere & Radius $r$ & $4\pi r^2$ & $\frac{4\pi}{3}r^3$\\
Cylinder & Height $h$, Radius $r$ & $2\pi r^2  + 2\pi rh$
& $\pi r^2 h$ \\
\hline\hline\\
\end{tabular}
\caption{Useful geometric formulas for common 3D shapes.}
\end{table}

\begin{table}[!h]
\begin{tabular}{cccccc}
\hline\hline\\
Particle & Symbol & Mass (kg) & Mass (eV/$c^2$) & Charge & Spin \\
\hline
Electron &$e^-$& $9.109\times10^{-31}$ & $5.11\times10^5$ 
  & $-1$ & $\frac{1}{2}$\\
Proton & $p$ & $1.673\times10^{-27}$ & $9.38\times10^8$ 
  & $+1$ & $\frac{1}{2}$\\
Neutron & $n$ &  $1.675\times10^{-27}$ & $9.40\times10^8$ 
  & $+1$ & $\frac{1}{2}$\\
Photon & $\gamma$ & $0$ & $0$ & $0$ & $0$\\
Electron Neutrino & $\nu_e$& $\approx 0$ & $< 2.5$ & $0$ & $\frac{1}{2}$\\
Muon Neutrino & $\nu_\mu$ & $< 3\times10^{-31}$ & $< 1.7\times10^5$ 
  & $0$ & $\frac{1}{2}$\\
Tau Neutrino & $\nu_\tau$ & $< 3\times10^{-29}$ & $< 1.8\times10^7$ 
  & $0$ & $\frac{1}{2}$\\
\hline\hline\\
\end{tabular}
\caption{Basic properties of common particles.  Charges are given
in multiples of the fundamental charge unit, $1.602\times10^{-19}$ coulombs.}
\end{table}

\begin{table}[!h]
\begin{tabular}{cccc}
\hline\hline\\
Object & Mass (kg) & Radius (km) & Distance (km) \\
\hline
Earth & $5.9736\times10^{24}$ & 6,400 & --- \\
Moon & $7.348\times10^{22}$ & 1738 & 384,400\\
Jupiter & $1.899\times10^{27}$ & $71,492$ & 620 to 920 million \\
Sun & $1.9891\times10^{30}$ & $6.960\times10^5$ & $149.6$ million \\
\hline\hline\\
\end{tabular}
\caption{Masses and distances from Earth for various astronomical bodies.}
\end{table}

\begin{table}[!h]
\begin{tabular}{cccc}
\hline\hline\\
Fuel & Main Product & $T$ (10$^9$ K) & Duration (yr)\\
\hline
H & He & 0.037 & $8.1\times10^6$\\
He & O, C & 0.19 & $1.2\times10^6$\\
C & Ne, Mg & 0.87 & $9.8\times10^2$\\
O & Si, S & 2.0 & $1.3$\\
Si & Fe & 3.3 & 0.031 (11 days)\\
\hline\hline\\
\end{tabular}
\caption{Nuclear ``burning'' stages for a star of 20 solar masses.
Source: ``Massive Star Evolution Through the Ages''
({\tt http://arxiv.org/abs/astro-ph/0211062}).}
\end{table}

\end{document}
