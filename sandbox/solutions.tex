\documentclass[12pt]{article}
\usepackage{amsmath,amssymb,fullpage,graphics,graphicx,yuniorlab}
\usepackage{problemset,url}

\begin{document}

\title{SNEWS Word Problems}
\author{}
\date{}
\maketitle

{\em Note to teachers:}  I have not been explicit with the
``significant figures'' in these calculations.  Astronomical
measurements are frequently imprecise:  we might be happy to measure a
star's mass to within five solar masses, for example.  In constructing
these problems, I have opted to stress concepts, rather than
arithmetic.  I would advise grading on a partial credit system, giving
several points to a student if he or she can manipulate the necessary
equation in the proper way, even if the arithmetic in the end has a
mistake or two.  If ``sig figs'' are a large part of your curriculum
(as I know they were in my day), you may wish to grade them more heavily.

\section{Exercise Solutions}

\begin{probdesc}{1}{How Big is a Neutrino Detector?}

\begin{enumerate}
\item[(a)] 
We can find the volume with a little dimensional analysis, a good
exercise in scientific notation:
\begin{eqnarray}
50{\rm\ kilotons} &=& \frac{10^6 {\rm\ kg}}{1 {\rm\ kiloton}}
  \times \frac{1 {\rm\ L}}{1 {\rm\ kg}}
  \times \frac{1 {\rm\ m}^3}{10^3 {\rm\ L}} \nonumber\\
&=& 5 \times 10^1 \times 10^6 \times 10^{-3} {\rm\ m}^3 \nonumber\\
&=& 5 \times 10^4 {\rm\ m}^3.\nonumber
\end{eqnarray}
``To a first approximation'', as the scientists say, we guess that the
tank is cubic.  Therefore, to find the length of its side, we take the
cube root:
\begin{equation}
\boxed{a = \sqrt[3]{5\times10^4 {\rm\ m}^3} \approx 36.8 {\rm\ m}.}
\end{equation}

\item[(b)]
\end{enumerate}

\end{probdesc}

\begin{probdesc}{2}{How Strong is the Signal?}
All the parts of this problem can be solved using the given equation
for intensity:
\begin{equation}
I(R) = I_0 \left(\frac{R_0}{R}\right)^2.
\end{equation}

\begin{enumerate}
\item[(a)] Here, we're told that $I_0 = 1 \frac{\rm J}{{\rm m}^2 \rm
  s}$, that $R_0 = 10$m and that we wish to know the intensity at $R =
  30$m.  Plug the distances we know into the equation:
\begin{eqnarray}
I(30 {\rm\ m}) &=& I_0 \left(\frac{10 \rm\ m}{30 \rm\ m}\right)^2
  \nonumber\\
&=& I_0 \left(\frac{1}{3}\right)^2 \nonumber\\
&=& \frac{I_0}{9} \nonumber\\
&=& \frac{1}{9}\frac{\rm J}{{\rm m}^2 \rm s},\nonumber
\end{eqnarray}
or in decimal form,
\begin{equation}
\boxed{I(30 {\rm\ m}) \approx 0.11 \frac{\rm J}{{\rm m}^2 \rm s}.}
\end{equation}


\item[(b)] This part may be a little trickier, because we're not told
  what $I_0$ is.  However, knowing algebra means we don't {\em have}
  to know $I_0$.  What we are told is that
\begin{equation}
I(R) = \frac{1}{100} I_0.
\end{equation}
We can plug this information into the intensity equation, like so:
\begin{equation}
I(R) = I_0\left(\frac{R_0}{R}\right)^2 = \frac{1}{100} I_0.
\end{equation}
Whatever $I_0$ is, it is surely a number greater than zero (otherwise,
the supernova would be invisible!).  Therefore, we can divide both
sides by the unknown quantity:
\begin{equation}
\left(\frac{R_0}{R}\right)^2 = \frac{1}{100}.
\end{equation}
Taking the square root of both sides, we find that
\begin{equation}
\frac{R_0}{R} = \frac{1}{10}.
\end{equation}
Cross-multiplying now shows that
\begin{equation}
\boxed{R = 10R_0 = 10 \rm\ megaparsecs.}
\end{equation}
\end{enumerate}
\end{probdesc}

\begin{probdesc}{3}{Temperature Scales}
The key formula here is
\begin{equation}
K = C + 273.15 = \frac{F + 459.67}{1.8}.
\end{equation}

\begin{enumerate}
\item[(a)] Well, there's no guarantee exactly what you find a
  comfortable room temperature, but lots of people seem to agree on
  around $72^\circ$F.  In kelvins, this is
\begin{equation}
\frac{72 + 459.67}{1.8} \approx 295.4 K.
\end{equation}
Subtract 273.15 $K$ to get the Celsius equivalent temperature, about
$22^\circ$C.

\item[(b)] Again, just subtract 273.15 $K$:
\begin{equation}
5500^\circ C - 273.15^\circ C = 5226.85\ K.
\end{equation}
\end{enumerate}

\end{probdesc}

\begin{probdesc}{4}{Stellar Masses}
Over the last several decades, scientists have discovered that the
mass of a star is the single most important factor in determining how
long it will ``live'' and in what fashion it will ``die''.
Frequently, masses for distant stars are given as multiples of our
Sun's mass.

\begin{enumerate}
\item[(a)] First, we multiply the mass of the sun (given in the data
  table) by 15:
\begin{equation}
15 M_{\rm Sun} \approx 2.98 \times 10^{31} {\rm\ kg}.
\end{equation}
This answers part (i); to solve part (ii), divide the previous result
by the mass of the Earth:
\begin{equation}
2.98\times 10^{31} {\rm kg} \times
\frac{1 M_{\rm Earth}}{5.9736\times10^{34} \rm\ kg}
\approx 5.0\times10^6 M_{\rm Earth}.
\end{equation}

\item[(b)] 70 Jupiter masses is about $1.33\times10^{29}$ kg.  Dividing
  this by $M_{\rm Sun}$ (the same value we used in part (a)) shows that
  70 Jupiter masses is about 0.067 solar masses.
\end{enumerate}
\end{probdesc}

\begin{probdesc}{5}{Signal Delay}
\end{probdesc}

\begin{probdesc}{6}{Nuclear Burning}
\end{probdesc}

\begin{probdesc}{7}{Half-Lives}
Here, all the problems can be solved using the half-life decay equation,
\begin{equation}
N(t) = N_0 \cdot \left(\frac{1}{2}\right)^{t / \tau}.
\end{equation}
It may be a useful hint to know that $N(t)$ and $N_0$ can be in any
units we wish:  grams, atoms, moles, etc.  If we measure $N_0$ in
grams, $N(t)$ will be in grams.  Likewise, as long as $t$ and $\tau$
are in the same units---seconds, years, or what have you---we don't
need to be picky about what units to use.  It is not necessary to
convert everything to seconds first!

\begin{enumerate}
\item[(a)] {\em Radium decay.}  We're told that the decay process for
  radium obeys the rule
  \begin{equation}
    ^{226}{\rm Ra} \rightarrow ^4{\rm He} + ^{222}{\rm Rn}.
  \end{equation}
  Here, $\tau = 1602$ years, and we're told that the time period $t$
  is 1000 years.  Any decent calculator can handle the tough work:
\begin{equation}
N(t) = 10 {\rm\ g} \cdot \left(\frac{1}{2}\right)^{\frac{1000}{1602}}
  \approx 6.49 {\rm\ g}.
\end{equation}
Radioactive decay ``burns up'' 3.51 g of radium, leaving 6.49 g
behind.  (Remember, this process takes a thousand years!)  A little
chemistry can tell us how much radon is released.  We know that every
atom of radium which decays releases one atom of radon, and the
Periodic Table informs us that one mole of radium has a mass of
226.0254 grams.  Furthermore, at ``standard temperature and pressure''
(273.15 $K$ and 1 atmosphere), one mole of any gas takes up 22.4
liters of space.  Using these facts, a little dimensional analysis
shows that
\begin{equation}
3.51 {\rm\ g\ Ra} \cdot
\frac{1 \rm\ mol\ Ra}{226.0254 \rm\ g\ Ra} \cdot
\frac{1 \rm\ mol\ Rn}{1 \rm\ mol\ Ra} \cdot
\frac{22.4 \rm\ L\ Rn}{1 \rm\ mol\ Rn} = 0.348 \rm\ L\ Rn.
\end{equation}


\item[(b)] {\em Carbon-14 dating.} Here, we're given that $\tau =
  5,730$ years.  The ``tricky'' part is that we don't know $N_0$ {\em
  or} $N(t)$, but (like the intensity problem earlier) we know their
  ratio.  The archaeologist measures that
\begin{equation}
N(t) = 0.80 N_0.
\end{equation}
Substituting this into the half-life equation, we find that
\begin{equation}
0.80 N_0 = N_0 \left(\frac{1}{2}\right)^{\frac{t}{\tau}}.
\end{equation}
Just like in the intensity problem, we can divide both sides by $N_0$:
\begin{equation}
0.80 = \left(\frac{1}{2}\right)^{\frac{t}{\tau}}.
\end{equation}
We can solve for $t/\tau$ using logarithms:
\begin{equation}
\frac{t}{\tau} = \log_{\frac{1}{2}} 0.80.
\end{equation}
Multiplying both sides by $\tau$ now shows that
\begin{equation}
t = \tau \log_{\frac{1}{2}} 0.80,
\end{equation}
which works out to about 1845 years.  This is a little {\em younger}
than the coins claimed to be, but remember that they could have been
put in the box when they were several years old already.

What about the experimental error?  The archaeologist could only
determine the carbon-14 content to within 2\%.  Supposing that the
real value were 78\%, then the box would be
\begin{equation}
t = \tau \log_{\frac{1}{2}} 0.80 = 2054
\end{equation}
years old.  Likewise, you can work out that if the real value were
82\%, the box would only be 1640 years old.  This gives an estimate of
the error involved---and an indication of how tricky getting accurate
measurements can be!

\item[(c)] {\em Supernova light curve.}  We're told that nickel-56 has
  a half-life of 6.077 days, and again we're given the ratio of $N(t)$
  and $N_0$.  This time,
\begin{equation}
N(t) = \frac{1}{64} N_0.
\end{equation}
Using the half-life equation,
\begin{equation}
\frac{1}{64} N_0 = N_0 \left(\frac{1}{2}\right)^{\frac{t}{\tau}},
\end{equation}
and dividing both sides by $N_0$,
\begin{equation}
\frac{1}{64} = \left(\frac{1}{2}\right)^{\frac{t}{\tau}}.
\end{equation}
Here, it's helpful to know the powers of two.  You can easily check
that $2^6 = 64$.  Turing this upside down, we see that
\begin{equation}
\left(\frac{1}{2}\right)^6 = \frac{1}{64}.
\end{equation}
Therefore, $\frac{t}{\tau} = 6$, and
\begin{equation}
\boxed{t \approx 36.5 \rm\ days.}
\end{equation}

\item[(d)] {\em Bonus.}  Since we just showed that in 36.5 days the
  nickel-56 would be down to $\frac{1}{64}$ of its original amount, if
  {\em all} the light came from nickel-56, then the supernova would
  only be $\frac{1}{64}$ as bright.  Observing that it remains half
  its original brightness after 80 days means that some other elements
  are responsible.
\end{enumerate}
\end{probdesc}

\begin{probdesc}{8}{Einstein's Equation}
According to Einstein's Special Theory of Relativity, matter and
energy are interchangeable.  It is possible to convert an amount of
mass into pure energy, which may take the form of light or other
electromagnetic radiation.  The exact rule is given by Einstein's
famous equation,
\begin{equation}
E = mc^2.
\end{equation}
Here, $c$ is the speed of light, roughly $3\times10^8$ meters per
second.  If $m$ is given in kilograms and $c$ in meters per second,
then $E$ will have units of joules.
\begin{enumerate}
\item[(a)] Here, $m = 1$ kg.  According to Einstein's equation, then,
\begin{equation}
\boxed{E = (1 {\rm\ kg}) \cdot \left(3\times10^8 \frac{\rm m}{\rm
    s}\right)^2 = 9\times10^{16} \rm J.}
\end{equation}


\item[(b)] Each photon should carry away half the energy of the
  original electron-positron pair, but since the electron and the
  positron have the same mass, each of the two photons should have the
  ``mass-energy'' of one electron.  Here, it's helpful to use the
  fourth column of the particle-data table, which gives masses in
  terms of eV/$c^2$.  The table tells us that an electron has a mass
  of $5.11\times10^5$ eV/$c^2$.  By Einstein's equation, we would
  multiply this by $c^2$ to get the energy, but this just cancels the
  $c^2$ already there!  Knowing that $1 {\rm MeV} = 10^6 {\rm
  eV}$, we can immediately say that the mass-energy of one electron is
  0.511~MeV.  Consequently, each photon has an energy of 0.511~MeV.

  You can find the same result by using the mass in kilograms and
  multiplying by $c^2 = 9\times10^{16} \frac{{\rm m}^2}{{\rm s}^2}$.
  Remember that $1 {\rm\ eV} = 1.602\times10^{-19} {\rm J}$.

\item[(c)] According to the quantum theory Max Planck helped found,
  the wavelength of a photon is inversely proportional to its energy.
  Typically, we use the Greek letter $\lambda$ ({\em lambda}) to stand
  for the wavelength.  Planck's equation says that
  \begin{equation}
    E = \frac{hc}{\lambda}
  \end{equation}
  where $h$ is {\em Planck's constant,} a number which experiments
  show is roughly $6.626\times10^{-34}$ joule-seconds.  There are (at
  least) two ways to solve this part: we can either plug the value of
  $E$ we found earlier into Planck's equation, or we can combine
  Planck's equation with Einstein's.  The two approaches should give
  identical answers.

  To demonstrate the latter approach, first set the two formulas for
  $E$ equal to each other:
  \begin{equation}
    mc^2 = \frac{hc}{\lambda}.
  \end{equation}
  We can multiply both sides by $\lambda$ to get
  \begin{equation}
    hc = \lambda m c^2,
  \end{equation}
  and we can divide both sides by $c$ to find
  \begin{equation}
    h = \lambda m c.
  \end{equation}
  Moving the $mc$ to the other side shows that
  \begin{equation}
    \lambda = \frac{h}{mc}.
  \end{equation}
  Plugging in all the values we've been given, the wavelength works
  out to
  \begin{equation}
    \boxed{\lambda \approx 2.42 \times 10^{-12} {\rm m}.}
  \end{equation}
  This is far too small see with the naked eye; in fact, it is a {\em
  gamma ray}, with a wavelength about 100 times smaller than the
  diameter of a hydrogen atom.


\item[(d)] Astrophysicists estimate that a Type II supernova can
  release $10^{44}$ joules of energy.  We can turn this into a mass
  measurement by using Einstein's equation in reverse:
  \begin{equation}
    m = \frac{E}{c^2} \approx 1.11\times10^{27} {\rm\ kg.}
  \end{equation}
  We suppose that the original star had a mass roughly twenty times
  that of the Sun:
  \begin{equation}
    M = 20 M_{\rm Sun} \approx 3.9782\times10^{31} {\rm\ kg.}
  \end{equation}
  This is much larger than the mass $m$ which went into making the
  supernova.  To see exactly how much larger, we compute the ratio:
  \begin{equation}
    \frac{m}{M} = 
    \frac{1.11\times10^{27} \rm\ kg}{3.9782\times10^{31} \rm\ kg}
    \approx 2.79\times10^{-5}.
  \end{equation}
  In everyday notation, the ratio is 0.0000279, or 0.00279\%.
\end{enumerate}
\end{probdesc}


\end{document}
