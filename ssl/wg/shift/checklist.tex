\documentclass{article}

\begin{document}

\title{SNEWS Shift Manual}         
\maketitle

\section{Shift Checks}      

Twice per day (morning and evening) perform the following checks
and make the corresponding entries in the online shift log page at\\
\texttt{https://snews.bnl.gov/wg/shift/}.

\begin{itemize}

\item Check that the server is running on bnlboom1:
  
\begin{verbatim}
bnlboom1:~> ps -u snalert
  PID TTY          TIME CMD
15976 ?        00:00:00 sh <defunct>
15984 ?        00:00:00 gcserv
15985 ?        00:00:00 gcserver.csh
15986 ?        00:00:00 gcserver
 6778 ?        00:00:00 sshd
 6781 pts/2    00:00:00 tcsh
\end{verbatim}
There should be three gcserver processes in memory, as listed above.

\item Check the server log file, \texttt{/home/snalert/gcalert/out/gcserver.out}.  Are there any unusual error messages or alarm records?
If any experiment is giving an unusually high rate of alarms, the
shifter should contact that experiment's subgroup member.

\item Note anything unusual in the online logbook, and contact
the relevant people if anything unusual is found.  

\item Send a ping message from your experiment's client machine, using
the \texttt{cping} utility, $e.g.$

\begin{verbatim}
cping all 0 0 0 1
\end{verbatim}

The arguments of \texttt{cping} are: server, ddmmyy, hhmmss, ns,
experiment.  If the first argument is ``all'', it will read
\texttt{serv\_ip\_addr.dat} (recommended).  Note that your working
directory must have this file (and must also have the client
certificates).
If ddmmyy, hhmmss, ns are all zero, it will send the
current UTC according to your
computer's clock (recommended).  Put your experiment's number as the last
argument.  To cause the server to dump the contents of the queue
to the log, add 100 to the experiment number.

After sending the ping, check that it showed up in the log. 

\item Now do the same for the backup server on boboom1.  Note that you 
can have \texttt{cping}
send to both servers at once by including both IPs in the 
\texttt{serv\_ip\_addr.dat} file.

%\item Check the eStar connection on bnlboom1 by running\\
%\texttt{/home/snalert/gcalert/estar\_check.pl};\\ go to
%\texttt{http://www.estar.org.uk/voevent/SNEWS/gov.bnl.snews/SNEWS/} to see
%if the test message was received (click on the link in the
%shift form).  If it was received correctly, there
%should be a file corresponding to the date and time of size $2.2$K.  If the
%file does not exist, or 
%has zero size, there may be a problem.  Email Alasdair Allan, 
%\texttt{aa@astro.ex.ac.uk}, in case of problems with the eStar server.


\item Check that your communications equipment is running,
batteries charged, etc.

\item If it is your last shift: send email reminder to the
next shiftworker.

\end{itemize}

\newpage

\section{Contact List}      


See \texttt{https://snews.bnl.gov/wg/contact\_list.html} for 
alert contact info.\\
\vspace{0.05in}

\noindent
\textbf{Alec Habig}

\noindent
Email: habig@neutrino.d.umn.edu\\
Office: 218-726-7214\\
Home: 218-724-9158\\
Cell: 218-290-4706\\
Soudan: 218-753-6611\\

\vspace{0.05in}

\noindent
\textbf{Kate Scholberg}

\noindent
Email: schol@phy.duke.edu\\
Office: 919-660-2962\\
Home: 919-402-1606 \\
Cell: 919-698-2620\\

\vspace{0.05in}

\noindent
\textbf{Brett Viren} (SNEWS sysadmin)

\noindent
Email: bv@bnl.gov\\
Office: 516-632-8085\\

\vspace{0.05in}

\noindent
\textbf{Eric McAlvin} (SNEWS sysadmin)

\noindent
Email: emcalvin@bnl.gov\\



\end{document}
